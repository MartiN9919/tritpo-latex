\documentclass[aspectratio=169, 12pt]{beamer}

\usepackage{lmodern}
\usepackage[utf8]{inputenc}
\usepackage[english,russian]{babel}
\usepackage{hyperref}

\hypersetup{
    colorlinks=true,
    linkcolor=blue,
    filecolor=blue,
    urlcolor=blue,
}

\setbeamertemplate{navigation symbols}{}

\usecolortheme{lily}
\usetheme{CambridgeUS}

\title{ТРиТПО}
\subtitle{Технологии разработки и тестирования программного обеспечения}
\author{Artsiom Vasilevich}
\institute[BSUIR]{Belarusian State University of Informatics and Radioelectronics}
\date{\tiny \today}

\begin{document}

\frame{\titlepage}

\begin{frame}[t]
    \frametitle{Введение в курс}
    \framesubtitle{Контакты}
    Email: \\
    \hspace{0.5cm} \href{mailto:avasilevich.work@gmail.com}{avasilevich.work@gmail.com} (prefered) \\
    \hspace{0.5cm} \href{mailto:a.vasilevich@bsuir.by}{a.vasilevich@bsuir.by} \\
    \vspace{\baselineskip}
    Telegram: \\
    \hspace{0.5cm} \href{https://t.me/avasilevich}{@avasilevich} \\
    \vspace{\baselineskip}
    Презентации: \\
    \hspace{0.5cm} \href{https://github.com/avasilevich/tritpo-latex}{github.com/avasilevich/tritpo-latex} \\
\end{frame}

\begin{frame}
    \frametitle{Введение в курс}
    \framesubtitle{Лабораторные занятия}
    \begin{itemize}
        \item Кол-во общих занятий --- ??
        \item Занятий в подгруппах --- достаточно чтобы сдать все Лр.
    \end{itemize}
\end{frame}

\begin{frame}
    \frametitle{Введение в курс}
    \framesubtitle{А <<ходить>> надо?} \pause
    \begin{itemize}
        \item Посещаете и сдаёте Лр. --- ОК \pause
        \item Посещаете и не сдаёте Лр. --- ОК (но зачем тогда ходить??) \pause
        \item Не посещаете и сдаёте Лр. --- ??? \pause
        \item Не посещаете и не сдаёте Лр. --- ОК
    \end{itemize}
\end{frame}

\begin{frame}
    \frametitle{Введение в курс}
    \framesubtitle{Путь к <<успеху>>}
    \begin{itemize}
        \item 6 лабораторных работ
        \item 1 --- 4 индивидуальные
        \item 5 --- 6 в парах
    \end{itemize}
\end{frame}

\begin{frame}
    \frametitle{Введение в курс}
    \framesubtitle{Система оценок}
    \begin{itemize}
        \item максимальная оценка за лр 1.0 \pause
        \item учёт ведётся с помощью установленных deadline \pause
        \begin{itemize}
            \item дата выдачи - start-date 10.09 \pause
            \item soft deadline для каждой из подгрупп 17.09 \pause
            \item hard deadline = start-date + $\sim$3 недели \pause
            \item лр сдана в soft deadline: $\leq$ 1.0
            \item лр сдана до hard но после soft: 0.2 --- 0.8
            \item лр сдана в hard или позже: $\leq$ 0.2
        \end{itemize}
    \end{itemize}
\end{frame}

\begin{frame}
    \frametitle{Введение в курс}
    \framesubtitle{Коротко о лабораторных} \pause
    \begin{enumerate}
        \item Java, JUnit, Git \pause
        \item Project SRS \pause
        \item Use-case, Activity, State diagrams \pause
        \item Class, Sequence, Component/Deployment diagrams \pause
        \item Project implementation + \textbf{patterns} + code-review \pause
        \item Test-cases and test-plan
    \end{enumerate}
\end{frame}

\begin{frame}[t]
    \frametitle{Лабораторная работа \textnumero 1}
    \framesubtitle{Порядок выполнения (1/2)} \pause
    \begin{itemize}
        \item Изучить теоретические сведения \pause
        \item Установить git (если Вы этого ещё не сделали) \pause
        \item Создать аккаунт на GitHub (если его ещё нет) с \textbf{настоящим} именем и номером группы \pause
        \item Fork repository \url{https://github.com/trtpo/laba1} \pause
        \item Clone repository --- git clone url \pause
        \item Собрать проект, запустить, проанализировать полученный результат
    \end{itemize}
\end{frame}

\begin{frame}[t]
    \frametitle{Лабораторная работа \textnumero 1}
    \framesubtitle{Порядок выполнения (2/2)}
    \begin{itemize}
        \item Внести изменения в код (каждое изменение – отдельный commit) \pause
        \begin{itemize}
            \item изменить цветовую гамму изображения \pause
            \item изменить уравнение фрактала \pause
            \item расширить набор операций над комплексными числами и использовать их в новом уравнении \pause
            \item добавить Unit тесты для проверки правильности новых операций \pause
        \end{itemize}
        \item Push изменений в origin \pause
        \item Создать pull-request
    \end{itemize}
\end{frame}

\begin{frame}[t]
    \frametitle{Лабораторная работа \textnumero 1}
    \framesubtitle{Вопросы} \pause
    \begin{enumerate}
        \item Понятие <<системы контроля версий>>
        \begin{itemize}
            \item Виды СКВ
            \item Git и особенности его работы
        \end{itemize} \pause
        \item Фазы производства ПО \pause
        \item Методологии разработки ПО
        \begin{itemize}
            \item Waterfall
            \item Спиральная
            \item Итеративная
            \item Инкрементная
            \item Гибкие методологии
        \end{itemize} \pause
        \item Java, JUnit
    \end{enumerate}
\end{frame}

\begin{frame}[t]
    \frametitle{Лабораторная работа \textnumero 1}
    \framesubtitle{Ресурсы}
    \begin{thebibliography}{10}
        \beamertemplatebookbibitems
        \bibitem{ProGit}
        Pro Git, Second Edition.
        \newblock {\em Open source comprehesive git book}.
        \newblock Scott Chacon, Ben Straub, Community, 2020.
        \bibitem{Java}
        Java: The Complete Reference, Eleventh Edition.
        \newblock {\em Book explains how to develop, compile, debug, and run Java programs}.
        \newblock Herbert Schildt, 2018.
        \bibitem{Lections}
        ЭУМКД ТРиТПО.
        \newblock {\em Лекционный материал по дисциплине ТРиТПО}.
        \newblock Наталья Искра, 2019.
    \end{thebibliography}
\end{frame}

\end{document}
